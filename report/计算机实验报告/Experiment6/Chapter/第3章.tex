\section{TinyShell的设计与实现}
\begin{center}
    总分45分
\end{center}

\subsection{设计}

\subsubsection{void eval(char *cmdline)函数(10分)}

\paragraph{函数功能:}执行命令行输入的命令,解析以空格分割的命令行参数,并构造最终会传递给execve的argv参数
\paragraph{参   数:}用户的命令行输入的命令
\paragraph{处理流程:}
\paragraph{要点分析:}

\subsubsection{int builtin\_cmd(char **argv)函数(5分)}

\paragraph{函数功能:}检查第一个命令是不是内置的外壳命令。这里所谓的内置外壳命令是quit,jobs,bg,fg。如果第一个不是内置命令,则返回0,否则执行内置命令,返回1
\paragraph{参   数:}Shell命令参数列表
\paragraph{处理流程:}如下图所示:
%    \digraph[scale=0.5]{builtin}{
%        randdir = LR;
%        node [shape="box",fontname = "Noto Sans CJK SC"];
%        "是否是内置命令" -> "执行内置命令"[label="是"];
%        "执行内置命令" -> "返回1";
%        "是否是内置命令" -> "返回0"[label="否"];
%    }
\paragraph{要点分析:}本函数比较简单,只要记得是内置命令的时候返回1即可。

\subsubsection{void do\_bgfg(char **argv) 函数(5分)}

\paragraph{函数功能:}执行内置的fg,bg命令
\paragraph{参   数:}Shell命令参数列表
\paragraph{处理流程:}
\paragraph{要点分析:}重点是对命令的合法性进行判断,判断是否是一个数字,并判断jobid的合法性,而后在对jobid进行操作。

\subsubsection{void waitfg(pid\_t pid) 函数(5分)}

\paragraph{函数功能:}等待处理PID,在此处是等待前台程序退出
\paragraph{参   数:}需要等待退出的子进程PID
\paragraph{处理流程:}
\paragraph{要点分析:}这里值得注意的是要注意在前台程序挂掉之前不能退出本函数,这里我使用了suspend函数用来进行线程同步。

\subsubsection{void sigchld\_handler(int sig) 函数(10分)}

\paragraph{函数功能:}处理SIGCHLD信号,这个信号处理程序回收子进程,如果子进程是僵尸进程,或者收到stop之类的信号时,进行调用
\paragraph{参   数:}被捕获到的信号值
\paragraph{处理流程:}
\paragraph{要点分析:}这里值得注意的问题就是要捕获子进程的终止信号,而后对子进程进行回收操作。注意在这里要打印Stop和Terminal信息,并对不同的信号进行处理。

\subsection{程序实现(tsh.c的全部内容)(10分)}
\inputminted{c}{../../../Experiment6/tsh.c}
