\section{计算机系统的基本信息获取编程}
\subsection{CPUID信息}
CPUID指令是intel IA32架构下获得CPU信息的汇编指令,可以得到CPU类型,型号,制造商信息,商标信息,序列号,缓存等一系列CPU相关的东西。
cpuid使用eax作为输入参数,eax,ebx,ecx,edx作为输出参数。

把eax = 0作为输入参数,cpuid指令执行以后,会返回一个12字符的制造商信息,前四个字符的ASCI码按低位到高位放在ebx,中间四个放在edx,最后四个字符放在ecx。
把eax = 1作为输入参数,cpuid指令执行以后,会返回CPU的版本、架构和签名,签名信息放在eax,特性放在edx和ecx,附加特性放在ebx。

\subsection{获取MAC}
使用\lstinline|ioctl|和\lstinline|socket|函数获取网卡MAC地址。

socket函数用于创建套接字,之后我们使用这个套接字来获取网卡的MAC地址,I/O控制函数ioctl用于对文件进行底层控制,这里的文件包含网卡、终端、磁带机、套接口等软硬件设施,实际的操作来自各个设备自己提供的ioctl接口。

\lstinline|int socket(int domain,int type, int protocol);|
\begin{itemize}
	\item domain参数:表示所使用的协议族(取值AF\_INET,表示采用internet协议族)
	\item type参数:表示套接口的类型(SOCK\_DGRAM,表示采用数据报类型套接口)
	\item protocol参数:表示所使用的协议族中某个特定的协议(自动选择,用0填充)
\end{itemize}

如果函数调用成功,套接口的描述符(非负整数)就作为函数的返回值,假如返回值为-1,就表明有错误发生。

\lstinline|int ioctl(int fd,int request,…)|。
\begin{itemize}
	\item fd参数:文件描述符(取套接口的描述符)
	\item request参数:指定信息获取码(SIONGIFHWADDR表示取硬件地址)
	\item 其后的request参数用于为实现I/O控制所必须传入或传出的参数
\end{itemize}

本实验需要用ifr结构传入网卡设备名,并传出6B的MAC地址。

\subsection{请提交源程序文件(10分)} 
\lstinputlisting[language = c]{Experiment1/sysinfo.c}