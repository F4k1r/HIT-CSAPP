\section{实验预习}
\subsection{画出存储器层级结构,标识容量价格速度等指标变化(5分)}

\begin{tikzpicture}
    \draw (0,0) -- (3.5,7) -- (7,0) -- (0,0);
    \draw (0.5,1) -- (6.5,1);
    \draw (1.0,2) -- (6.0,2);
    \draw (1.5,3) -- (5.5,3);
    \draw (2.0,4) -- (5.0,4);
    \draw (2.5,5) -- (4.5,5);
    \draw (3.0,6) -- (4.0,6);
    \draw (3.5,0.5) node {远程二级存储};
    \draw (3.5,1.5) node {本地二级存储};
    \draw (3.5,2.5) node {主存(DRAM)};
    \draw (3.5,3.5) node {L3};
    \draw (3.5,4.5) node {L2};
    \draw (3.5,5.5) node {L1};
    \draw (3.5,6.2) node {Reg};
    \draw [->] (-3.0, 0) -- (-3.0,7);
    \draw (-2.0, 5) node {更小};
    \draw (-2.0, 4) node {更快};
    \draw (-2.0, 3) node {更贵};
\end{tikzpicture}

\subsection{用CPUZ等查看你的计算机Cache各参数,写出各级Cache的C S E B s e b(5分)}

\begin{tabular}{|c|c|c|c|c|c|c|c|}
    \hline 
    & C & S & E & B & s & e & b \\ 
    \hline 
    一级数据缓存 & 32kb & 64 & 8 & 64 & 6 & 3 & 6 \\ 
    \hline 
    一级指令缓存 & 32kb & 64 & 8 & 64 & 6 & 3 & 6 \\ 
    \hline 
    二级缓存 & 256kb & 64 & 64 & 512 & 6 & 2 & 9 \\ 
    \hline 
    三级缓存 & 6144kb & 64 & 64 & 3072 & 6 & $2+\log_23$ & $10+\log_23$ \\ 
    \hline 
\end{tabular} 

\subsection{写出各类Cache的读策略与写策略(5分)}

\subsection{写出用gprof进行性能分析的方法(5分)}

\subsection{写出用Valgrind进行性能分析的方法(5分)}