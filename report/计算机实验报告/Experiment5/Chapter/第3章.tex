\section{Cache模拟与测试}
\begin{center}
    每个用例的每一指标5分(最后一个用例10分)\\
    与参考csim-ref模拟器输出指标相同则判为正确
\end{center}

\subsection{Cache模拟器设计}

% 提交csim.c

\subsubsection{程序设计思想}

\subsubsection{输出截图}

\begin{figure}[H]
    \centering
    \includegraphics[width=0.7\linewidth]{figures/CSim_1}
    \caption{测试用例1的输出截图(5分)}
    \label{fig:csim1}
\end{figure}

\begin{figure}[H]
    \centering
    \includegraphics[width=0.7\linewidth]{figures/CSim_2}
    \caption{测试用例2的输出截图(5分)}
    \label{fig:csim2}
\end{figure}

\begin{figure}[H]
    \centering
    \includegraphics[width=0.7\linewidth]{figures/CSim_3}
    \caption{测试用例3的输出截图(5分)}
    \label{fig:csim3}
\end{figure}

\begin{figure}[H]
    \centering
    \includegraphics[width=0.7\linewidth]{figures/CSim_4}
    \caption{测试用例4的输出截图(5分)}
    \label{fig:csim4}
\end{figure}

\begin{figure}[H]
    \centering
    \includegraphics[width=0.7\linewidth]{figures/CSim_5}
    \caption{测试用例5的输出截图(5分)}
    \label{fig:csim5}
\end{figure}

\begin{figure}[H]
    \centering
    \includegraphics[width=0.7\linewidth]{figures/CSim_6}
    \caption{测试用例6的输出截图(5分)}
    \label{fig:csim6}
\end{figure}

\begin{figure}[H]
    \centering
    \includegraphics[width=0.7\linewidth]{figures/CSim_7}
    \caption{测试用例7的输出截图(5分)}
    \label{fig:csim7}
\end{figure}

\begin{figure}[H]
    \centering
    \includegraphics[width=0.7\linewidth]{figures/CSim_8}
    \caption{测试用例8的输出截图(10分)}
    \label{fig:csim8}
\end{figure}

\subsection{矩阵转置设计}

% 提交trans.c
% 提交csim.c

\subsubsection{程序设计思想}

\subsubsection{输出截图}

%32×32(10分):运行结果截图
%64×64(10分):运行结果截图
%61×67(20分):运行结果截图