\section{实验基本信息}

\subsection{实验目的}
\begin{enumerate}
	\item 运用现代工具进行计算机软硬件系统的观察与分析
	\item 运用现代工具进行Linux下C语言的编程调试
	\item 初步掌握计算机系统的基本知识与各种类型的数据表示
\end{enumerate}

\subsection{实验环境与工具}

\subsubsection{硬件环境}
\begin{itemize}
	\item CPU: Intel Core i7-6700HQ @ 8x 3.5GHz
	\item RAM: 3516MiB / 7899MiB
\end{itemize}

\subsubsection{软件环境}
\begin{itemize}
	\item Manjaro Linux

\end{itemize}

\subsubsection{开发工具}
\begin{itemize}
	\item Virtual BOX
	\item GCC
	\item Clion
\end{itemize}

\subsection{实验预习}

\begin{itemize}
	\item 在Windows下编写hellowin.c,显示“Hello\ 1160300202冯云龙” %可用VI、VIM、EMACS、GEDIT等,换成学生自己信息
	\item 在Linux下编写hellolinux.c,显示“Hello\ 1160300202冯云龙” %可用VI、VIM、EMACS、GEDIT等,换成学生自己信息
	\item 编写showbyte.c,以16进制显示文件hello.c等的内容:每行16个字符,上一行为字符,下一行为其对应的16进制形式。
	\item 编写datatype.c,定义C所有类型的全局变量,并赋初值。%如整数可以是学号(数字部分),字符串可以是你的姓名,浮点数可以是身份证号的数字部分。主程序打印每个变量的变量名、变量值、变量地址、变量对应16进制的内存各字节。
\end{itemize}