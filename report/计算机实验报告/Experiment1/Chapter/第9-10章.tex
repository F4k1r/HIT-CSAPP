\section{程序运行分析}
\subsection{SUM的分析(20分)}

\begin{lstlisting}[language = c]
int sum(int a[], unsigned len) {
	int i, sum = 0;
	for (i = 0; i <= len - 1; i++)
	sum += a[i];
	return sum;
}
\end{lstlisting}

\subsubsection{运行结果}程序运行后会发生数组越界,而后被系统强行终止。

\subsubsection{原因分析}\lstinline[language=c]|len|是\lstinline[language=c]|unsigned|类型,输入\lstinline[language=c]|len=0|,导致的结果其实是\lstinline[language=c]|i|始终与\lstinline[language=c]|0xFFFFFFFF|比较大小,\lstinline[language=c]|i|是\lstinline[language=c]|int|类型,进行了类型后转换后始终比\lstinline[language=c]|len|小,之后在循环过程中就会越来越大,最终数组访问到无权限的位置被强行终止。

\subsubsection{改进方法}修改 \lstinline|for(i = 0;i <= len-1;i++)| 为 \lstinline|for(i = len;i > 0;i++)|。

\subsection{FLOAT的分析(20分)} 
\begin{lstlisting}[language = c]
#include <stdio.h>

int main() {
  float f;
  for (;;) {
  printf("请输入一个浮点数:");
  scanf("%f", &f);
  printf("这个浮点数的值是:%f\n", f);
  if (f == 0)
    break;
  }
  return 0;
}
\end{lstlisting}

\subsubsection{运行结果}
\begin{tabular}{|c|c|c|c|}
	\hline 
	输入 & 输出 & 输入 & 输出 \\ 
	\hline 
	61.419997 & 61.419998 & 10.186810 & 10.186810 \\ 
	\hline 
	61.419998 & 61.419998 & 10.186811 & 10.186811 \\ 
	\hline 
	61.419999 & 61.419998 & 10.186812 & 10.186812 \\ 
	\hline 
	61.420000 & 61.419998 & 10.186813 & 10.186813 \\ 
	\hline 
	61.420001 & 61.420002 & 10.186814 & 10.186814 \\ 
	\hline
	          &           & 10.186815 & 10.186815 \\ 
	\hline 
\end{tabular} 

\subsubsection{原因分析}
由于\lstinline[language=c]|float|是用有限的内存存储无限的数据,这当然是不可能的,所以float的值是离散的,不精确的,在输入60.419998等数据时,精度是不足的,C语言对其进行了舍入,而在输入10.186810时,\lstinline[language=c]|float|精度足够,所以可以取得精确值。

\subsubsection{注意事项}
C语言中\lstinline[language=c]|float|的精度是有限的,数值越大,越不精确,所以在使用\lstinline[language=c]|float|类型时需要考虑精确度的问题。


\section{总结}
\subsection{本次实验的收获}
本次实验,我学会了虚拟机的安装与使用,学会使用C语言和操作系统进行交互,而不仅仅停留于表面,在对程序进行分析的时候,更进一步的明白了C语言的相关特性,获益匪浅。

\subsection{对本次实验内容的建议} 
希望能同时给出 \LaTeX 模板,方便使用。


