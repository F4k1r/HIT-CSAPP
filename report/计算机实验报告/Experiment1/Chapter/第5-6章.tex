\section{以16进制查看程序HELLO.C}
\subsection{请查看HELLOWIN.C与HELLOLINUX.C的编码(3分)}

HelloWin.c采用\dlmu[2cm]{GBK}编码,HelloLinux.c 采用\dlmu{UTF-8}编码,你的姓名\dlmu[2cm]{冯云龙}分别编码为:\dlmu[7cm]{B7\ EB\ D4\ C6\ C1\ FA}\\
与\dlmu[7cm]{E5\ 86\ AF\ E4\ BA\ 91\ E9\ BE\ 99\ 0A}。

HelloWin.c在Linux下用gcc缺省模式编译后运行结果为:\dlmu[5cm]{Hello\ 1160300202\ ??????}。

\subsection{请查看HELLOWIN.C与HELLOLINUX.C的回车(3分)} 

Windows下的回车编码为:\dlmu[2cm]{0D\ 0A},Linux下的回车编码为:\dlmu[2cm]{0A 0A}。
交叉打开文件的效果是\dlmu[6cm]{hellolinux.c全部变成一行},
\dlmu[6cm]{hellowin.c 后面多了一个 $\hat{ } M$ 符号}。

 \section{程序的生成CPP、GCC、AS、LD}
\subsection{请提交每步生成的文件(4分)}
\embeddedfile[Source file]{hellolinux.c}{./Experiment1/hello.c}
\embeddedfile[Preprocess file]{hellolinux.i}{./Experiment1/hello.o}
\embeddedfile[Assemble file]{hellolinux.o}{./Experiment1/hello.c}
\embeddedfile[Object file]{hellolinux.out}{./Experiment1/hello.out}


\begin{enumerate}
	\item 预处理,生成预编译文件(.i文件):\lstinline|Gcc –E hello.c –o hello.i|
	\item 编译,生成汇编代码(.s文件):\lstinline|Gcc –S hello.i –o hello.s|
	\item 汇编,生成目标文件(.o文件):\lstinline|Gcc –c hello.s –o hello.o|
	\item 链接,生成可执行文件:\lstinline|Gcc hello.o –o hello.out|
\end{enumerate}

\lstinputlisting[language = c]{Experiment1/hello.c}

需要提交的文件在PDF附件当中。