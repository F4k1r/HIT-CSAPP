\section{实验预习}
\subsection{请按照入栈顺序,写出C语言32位环境下的栈帧结构(5分)}

\begin{tikzpicture}
    \draw (0,0) rectangle (5,10);
    \draw (0,9) -- (5,9);
    \draw (6,8.5) node {当前帧};
    \draw (-1,9.5) node {栈指针\%esp};
    \draw (2.5,9.5) node {参数构造区域};
    \draw (0,8) -- (5,8);
    \draw (-1,8.5) node {-4};
    \draw (2.5,8.5) node {寄存器值,局部和临时变量};
    \draw (0,7) -- (6,7);
    \draw (-1,7.5) node {帧指针\%ebp};
    \draw (2.5,7.5) node {被保存的\%ebp};
    \draw (0,6) -- (5,6);
    \draw (6,4.5) node {调用者帧};
    \draw (-1,6.5) node {+4};
    \draw (2.5,6.5) node {返回地址};
    \draw (0,5) -- (5,5);
    \draw (-1,5.5) node {+8};
    \draw (2.5,5.5) node {参数1};
    \draw (0,4) -- (5,4);
    \draw (2.5,4.5) node {...};
    \draw (0,3) -- (5,3);
    \draw (-1,3.5) node {+4+4n};
    \draw (2.5,3.5) node {参数n};
    \draw (0,2) -- (6,2);
    \draw (2.5,2.5) node {...};
    \draw (6,1) node {较早的帧};
    \draw (2.5,1) node {...};
    \draw [->] (-4,4) -- (-4,6) node {地址减小} -- (-4,8);
\end{tikzpicture}

IA32程序使用栈来支持过程调用。机器用栈来传递过程参数、存储返回信息、保存寄存器用于以后恢复,以及本地存储。为单个过程分配的那部分叫做栈帧。如上图所示,栈帧的最顶端以两个指针为分界线,寄存器\%ebp为帧指针,寄存器\%esp为栈指针。当程序执行时,栈指针可以移动,因此大多数信息的访问都是相对于栈指针的。

假设过程P(调用者)调用过程Q(被调用者),Q的参数放在P的栈帧中。另外,当P调用Q时,P中的返回地址被压入栈中,形成P的栈帧末尾。返回地址就是当程序从Q返回时应该继续执行的地方。Q的栈帧从保存帧指针(\%esp)开始,后面是保存的其他寄存器的值。

过程Q也用栈来保存其他不能存放在寄存器中的局部变量,这样做的主要原因是:
\begin{itemize}
    \item 没有足够多的寄存器存放所有的局部变量。
    \item 有些局部变量是数组、结构体或者如C++中的对象,因此必须通过数组,结构体或者对象引用来访问。
    \item 有些局部变量是左值,如果用户代码对一个左值进行了取地址操作,就必须得为这样的局部变量生成一个地址。
\end{itemize}

另外,Q会用栈帧来存放它调用的其他过程的参数,即函数的实参。如上图所示,在被调用的时候,第一个参数存放在相对于\%ebp偏移量为8的位置,剩下的参数(假设参数数据类型长度为4字节)存储在后续的4字节内存中,所以参数i就在相对于\%ebp偏移量为4+4i的地方。较大的参数(如结构体或者对象)需要更大的栈上空间。

\subsection{请按照入栈顺序,写出C语言64位环境下的栈帧结构(5分)}

\begin{tikzpicture}
\draw (0,0) rectangle (5,13);
\draw (6,11.5) node {红区};
\draw (2.5,11.5) node {*red zone* 128 bytes};
\draw (0,10) -- (6,10);
\draw (0,9) -- (5,9);
\draw (6,8.5) node {当前帧};
\draw (-1,9.5) node {栈指针\%rsp};
\draw (2.5,9.5) node {参数构造区域};
\draw (0,8) -- (5,8);
\draw (-1,8.5) node {-8};
\draw (2.5,8.5) node {寄存器值,局部和临时变量};
\draw (0,7) -- (6,7);
\draw (-1,7.5) node {帧指针\%rbp};
\draw (2.5,7.5) node {被保存的\%rbp};
\draw (0,6) -- (5,6);
\draw (6,4.5) node {调用者帧};
\draw (-1,6.5) node {+8};
\draw (2.5,6.5) node {返回地址};
\draw (0,5) -- (5,5);
\draw (-1,5.5) node {+16};
\draw (2.5,5.5) node {参数1};
\draw (0,4) -- (5,4);
\draw (2.5,4.5) node {...};
\draw (0,3) -- (5,3);
\draw (-1,3.5) node {+8+8n};
\draw (2.5,3.5) node {参数n};
\draw (0,2) -- (6,2);
\draw (2.5,2.5) node {...};
\draw (6,1) node {较早的帧};
\draw (2.5,1) node {...};
\draw [->] (-4,4) -- (-4,6) node {地址减小} -- (-4,8);
\end{tikzpicture}

相对于X86来说有了以下变化。
\begin{itemize}
    \item 没有帧指针。寄存器\%rbp变为了普通的通用目的寄存器。作为代替,对栈位置的引用相对于栈指针。大多数函数在调用开始时分配所需的整个栈存储空间,并保持栈指针指向固定的位置。
    \item 由于增加了通用目的寄存器的缘故,许多函数不需要栈帧。只有那些不能将所有的局部变量都存放在寄存器中的函数才需要在栈上分配空间。
    \item 函数传参的实参(最多是前6个)通过寄存器传递,而不是在栈上。这消除了在栈上存储和检索值的开销。
    \item 函数最多可以访问超过当前栈指针\%rsp值128个字节的栈上空间(地址低于当前栈指针的值)。这就允许了一些函数将信息存储在栈上而无需修改栈指针。
    \item callq指令将一个64位的返回地址压栈,而不是32位。
\end{itemize}

\subsection{请简述缓冲区溢出的原理及危害(5分)}
\paragraph{原理}在计算机内部,输入数据通常被存放在一个临时空间内,这个临时存放的空间就被称为缓冲区,缓冲区的长度事先已经被程序或者操作系统定义好了。向缓冲区内填充数据,如果数据的长度很长,超过了缓冲区本身的容量,那么数据就会溢出存储空间,而这些溢出的数据还会覆盖在合法的数据上,这就是缓冲区和缓冲区溢出的道理。

\paragraph{危害}利用缓冲区溢出攻击,可以导致程序运行失败、系统当机、重新启动等后果。更为严重的是,可以利用它执行非授权指令,甚至可以取得系统特权,进而进行各种非法操作。

\subsection{请简述缓冲器溢出漏洞的攻击方法(5分)}
通过往程序的缓冲区写超出其长度的内容,造成缓冲区的溢出,从而破坏程序的堆栈,使程序转而执行其它指令,以达到攻击的目的。

\begin{enumerate}
    \item 在程序的地址空间里安排适当的代码。
    \item 通过适当的初始化寄存器和内存,让程序跳转到入侵者安排的地址空间执行。
\end{enumerate}

\subsection{请简述缓冲器溢出漏洞的防范方法(5分)}

\begin{enumerate}
    \item 通过使被攻击程序的数据段地址空间不可执行,从而使得攻击者不可能执行被植入被攻击程序输入缓冲区的代码,这种技术被称为非执行的缓冲区技术。
    \item 编写正确的代码。
\end{enumerate}

